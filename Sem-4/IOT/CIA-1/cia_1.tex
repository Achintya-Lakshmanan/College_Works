\documentclass{article}
\usepackage[utf8]{inputenc}
\usepackage{geometry}
 \geometry{
 a4paper,
 total={170mm,257mm},
 left=20mm,
 top=20mm,
 }
 \usepackage{graphicx}
 \usepackage{titling}

 \title{Usage of Edge Computing to Create Smart Grids Based on Internet of Things}
\author{Achintya Lakshmanan}
\date{21 Jan 2023}
 
 \usepackage{fancyhdr}
\fancypagestyle{plain}{%  the preset of fancyhdr 
    \fancyhf{} % clear all header and footer fields
    \fancyfoot[R]{\includegraphics[width=2cm]{snu_logo.jpg}}
    \fancyfoot[L]{\thedate}
    \fancyhead[L]{CIA - 1 Assignment}
    \fancyhead[R]{\theauthor}
}

\makeatletter
\def\@maketitle{%
  \newpage
  \null
  \vskip 1em%
  \begin{center}%
  \let \footnote \thanks
    {\LARGE \@title \par}%
    \vskip 1em%
    %{\large \@date}%
  \end{center}%
  \par
  \vskip 1em}
\makeatother
\usepackage{cmbright}

\begin{document}

\maketitle

\noindent\begin{tabular}{@{}ll}
    Student & \theauthor\\
    Class & AI DS - A\\
    Reg No & 21011101064\\
     Link &  https://ieeexplore.ieee.org/document/8727940
\end{tabular}

\section{Summary}
The "Internet of Things Based Smart Grids Supported by Intelligent Edge Computing" research paper is focused on exploring the ways in which Internet of Things (IoT) technology can be integrated with smart grids to improve their efficiency and reliability. The paper argues that by leveraging the capabilities of IoT devices, smart grids can gain access to a large amount of real-time data that can be used to optimize energy distribution and management.

The paper also proposes the use of intelligent edge computing as a way to support the integration of IoT devices in smart grids. Edge computing allows for the processing and analysis of data at the edge of the network, rather than in a centralized location. This approach is beneficial for smart grids as it reduces the amount of data that needs to be transmitted to a central location, reduces the load on the network, and enables faster decision-making.

The paper discussed that the integration of IoT technology and intelligent edge computing can provide several benefits to smart grids such as:

\begin{itemize} 
\item An improvement to energy management by using real-time data from IOT devices to optimize energy distribution and management

\item An increase to the capacity of renewable energy sources through the use of IOT devices to monitor and control renewable energy sources by the use of solar panels and wind turbines.

\item A better understanding of energy consumption patterns, by using IoT devices to monitor energy consumption in real-time.

\item Reduction in energy consumption and costs, by the use of IoT devices to monitor and control appliances and other equipment.
\end{itemize}

However, the paper also highlights the challenges that need to be addressed in order to successfully integrate IoT technology and intelligent edge computing into smart grids. These challenges include:

\begin{itemize}
\item Security and privacy concerns, as the integration of IoT devices in smart grids can create new vulnerabilities that need to be addressed.

\item Usage of IOT devices can lead to a large amount of data that needs to be stored, analyzed, and used to make decisions which will require a Robust data management and analytics systems.

\item Integration of various systems and platforms will be needed.
\end{itemize}

In conclusion, the research paper argues that the integration of IoT technology and intelligent edge computing can significantly improve the performance of smart grids and support the transition to more sustainable energy systems. By leveraging the capabilities of IoT devices and intelligent edge computing, smart grids can gain access to a large amount of real-time data that can be used to optimize energy distribution and management, reduce energy consumption and costs and to address the challenges that come with the integration of IoT technology and intelligent edge computing in smart grids.

\section{Key Contributions}
\begin{itemize}

\item The paper provides a comprehensive overview of the key components of an IoT-based smart grid system, including IoT devices, such as smart meters, and advanced communication technologies, such as low-power wide-area networks (LPWANs) and 5G networks, that enable the collection of real-time data from various parts of the power grid.

\item They review the various challenges and limitations of IoT-based smart grid systems, including security, privacy, and scalability issues. The papers also provides recommendations for addressing these challenges, such as using blockchain technology for secure and private data storage and communication.

\item The paper presents a comprehensive and up-to-date review of the current state of IoT-based smart grid systems and their potential for improving the efficiency, reliability, and sustainability of power grid systems.
\end{itemize}

\section{My Views}
IoT-based smart grids have the potential to provide real-time monitoring and control of the grid, increased integration of renewable energy sources, and advanced demand response capabilities. However, there are also challenges that need to be addressed, such as security, privacy, and scalability issues. The integration of AI, machine learning and blockchain technology can help to address these challenges and improve the overall performance of the smart grid system. Standardization efforts also play a crucial role in ensuring the interoperability and security of smart grid systems. Overall, IoT-based smart grids have the potential to significantly improve the functioning of power grid systems and support the transition to a more sustainable energy future.

\section{Agreements}
\begin{itemize}

\item IoT-based smart grids have the potential to improve the efficiency, reliability, and sustainability of power grid systems by providing real-time monitoring and control of the grid. This can help to reduce energy loss and improve the overall performance of the grid.

\item Security and privacy are important considerations in the design and implementation of IoT-based smart grid systems. This can help to reduce dependence on fossil fuels and support the transition to a more sustainable energy future.

\item Standardization efforts for IoT-based smart grid systems can help to ensure interoperability and security of smart grid systems. This can help to reduce energy costs and improve the overall performance of the grid. This can help to ensure that different devices and systems can work together seamlessly, improving the overall performance of the smart grid.

\item Integration of AI, machine learning and blockchain technology can help to address the challenges and improve the overall performance of the smart grid system. These technologies can help to improve forecasting of energy demand, optimize energy 

\end{itemize}

\section{Pitfalls}

\begin{itemize}

\item The paper does not address the specific challenges and limitations of deploying IoT-based smart grid systems in different geographical locations or under different regulatory frameworks. It can be argued that different regions have different energy demands, infrastructure, and regulations, and that these factors need to be considered when deploying smart grid systems.

\item The paper does not provide enough information on the cost-benefit analysis of IoT-based smart grid systems. Without a clear understanding of the costs and benefits of the systems, it's difficult to evaluate the economic feasibility of implementation.

\end{itemize}

\end{document}
